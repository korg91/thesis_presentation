\documentclass{beamer}

\uselanguage{italian}
\languagepath{italian}
\deftranslation[to=italian]{Theorem}{Teorema}
\deftranslation[to=italian]{Definition}{Definizione}
\deftranslation[to=italian]{Corollary}{Corollario}

\usetheme[progressbar=frametitle, block=fill]{metropolis} % Use metropolis theme

\usepackage[utf8]{inputenc}
\usepackage{forest}
\usepackage{xspace}
\usepackage{xcolor}

%%% inizio comandi per stile per teoremi: "numero. Titolo" %%%
\newtheoremstyle{num.custom-title}
  {\topsep}   % ABOVESPACE
  {\topsep}   % BELOWSPACE
  {\itshape}  % BODYFONT
  {0pt}       % INDENT (empty value is the same as 0pt)
  {\bfseries} % HEADFONT
  {}         % HEADPUNCT
  {5pt plus 1pt minus 1pt} % HEADSPACE
  {\thmnumber{#2.}\thmnote{ #3}}
  
\theoremstyle{num.custom-title}  
\newtheorem{teo_custom-title}[theorem]{} % per usarlo basta \begin{teo_custom-title}[<Titolo teorema>] (usa automaticamente la numerazione di [teo])
%%% fine comandi per stile per teoremi: "numero. Titolo" %%%

%%% inizio comandi per stile per teoremi: "Titolo" %%%
\newtheoremstyle{custom-title}{}{}{\normalfont}{}{\bfseries}{.}{.5em}{\thmnote{#3}#1}
\theoremstyle{custom-title}
\newtheorem*{teo_custom-title_nonum}{}
%%% fine comandi per stile per teoremi: "numero. Titolo" %%%

\newenvironment{claim}[1]{\par\noindent\underline{Claim#1:}\space}{} %per i claim
\newenvironment{claimproof}[1]{\par\noindent\underline{Proof:}\space#1}{\leavevmode\unskip\penalty9999 \hbox{}\nobreak\hfill\quad\hbox{$\blacksquare$}} %per le dimostrazioni dei claim

\DeclareMathOperator{\dom}{dom}
\DeclareMathOperator{\ran}{ran}
\DeclareMathOperator{\orb}{orb}
\DeclareMathOperator{\id}{id}
\DeclareMathOperator{\rk}{rk}
\DeclareMathOperator{\tor}{tor}
\let\o\relax % elimina \o dai comandi già definiti
\DeclareMathOperator{\o}{\mathsf{o}}
\let\Im\relax % elimina \o dai comandi già definiti
\DeclareMathOperator{\Im}{Im}
\DeclareMathOperator{\Zdv}{Zdv}
\DeclareMathOperator{\Hom}{Hom}
\DeclareMathOperator{\End}{End}
\DeclareMathOperator{\Ann}{Ann}
\DeclareMathOperator{\E}{\mathbb{E}}
\DeclareMathOperator{\PP}{\mathcal{P}}
\DeclareMathOperator{\LL}{\mathcal{L}}
\DeclareMathOperator{\Hrtg}{\text{Hrtg}}
\DeclareMathOperator{\J}{\mathcal{J}}
\DeclareMathOperator{\Z}{\mathbb{Z}}
\DeclareMathOperator{\U}{\mathfrak{U}}
\DeclareMathOperator{\PPP}{\mathbb{P}}
\DeclareMathOperator{\V}{\mathcal{V}}
\DeclareMathOperator{\Var}{Var}
\DeclareMathOperator{\Cov}{Cov}
\DeclareMathOperator{\a01}{\{0,1\}^{\star}}
\DeclareMathOperator{\imp}{\Rightarrow}
\DeclareMathOperator{\pmi}{\Leftarrow}
\DeclareMathOperator{\Pic}{Pic}
\DeclareMathOperator{\sm}{\setminus}
\DeclareMathOperator{\sse}{\subseteq}
\DeclareMathOperator{\cl}{cl}
\DeclareMathOperator{\Spec}{Spec}
\DeclareMathOperator{\Tr}{Tr}
\DeclareMathOperator{\spn}{span}
\DeclareMathOperator{\q}{\mathsf{q}}
\DeclareMathOperator{\h}{h}
\DeclareMathOperator{\GL}{GL}
\DeclareMathOperator{\type}{type}
\DeclareMathOperator{\height}{height}
\DeclareMathOperator{\length}{length}
\DeclareMathOperator{\restr}{\upharpoonright}
\DeclareMathOperator{\down}{\downarrow}
\DeclareMathOperator{\up}{\uparrow}
\DeclareMathOperator{\cf}{cf}
\DeclareMathOperator{\mos}{mos}
\DeclareMathOperator{\trcl}{trcl}
\DeclareMathOperator{\Fn}{Fn}
%\DeclareMathOperator{\conc}{^\frown}
%\DeclareMathOperator{\gcd}{GCD}


\newcommand{\AC}{\ensuremath{\mathsf{AC}}\xspace}
\newcommand{\CC}{\ensuremath{\mathsf{CC}}\xspace}
\newcommand{\DC}{\ensuremath{\mathsf{DC}}\xspace}
\newcommand{\ZF}{\ensuremath{\mathsf{ZF}}\xspace}
\newcommand{\ZFC}{\ensuremath{\mathsf{ZFC}}\xspace}
\newcommand{\LS}{\ensuremath{\mathsf{LS}}\xspace}
\newcommand{\AMC}{\ensuremath{\mathsf{AMC}}\xspace}
\newcommand{\TP}{\ensuremath{\mathsf{TP}}\xspace}
\newcommand{\GCH}{\ensuremath{\mathsf{GCH}}\xspace}
\newcommand{\CH}{\ensuremath{\mathsf{CH}}\xspace}
\newcommand{\SH}{\ensuremath{\mathsf{SH}}\xspace}
\newcommand{\nSH}{\ensuremath{\neg\mathsf{SH}}\xspace}
\newcommand{\MA}{\ensuremath{\mathsf{MA}}\xspace}
\newcommand{\ST}{\ensuremath{\mathsf{ST}}\xspace}
\newcommand{\KT}{\ensuremath{\mathsf{KT}}\xspace}
\newcommand{\KH}{\ensuremath{\mathsf{KH}}\xspace}
\newcommand{\HRule}{\rule{\linewidth}{0.5mm}} %per la prima pagina
\newcommand{\qedblack}{\hfill $\blacksquare$}
\newcommand{\ol}{\overline}
\newcommand{\ul}{\underline}
\newcommand{\A}{\mathcal{A}}
\newcommand{\B}{\mathcal{B}}
\newcommand{\C}{\mathbb{C}}
\newcommand{\F}{\mathcal{F}}
\newcommand{\I}{\mathcal{I}}
\newcommand{\M}{\mathcal{M}}
\newcommand{\Q}{\mathbb{Q}}
\newcommand{\N}{\mathbb{N}}
\newcommand{\R}{\mathbb{R}}
\newcommand{\G}{\mathcal{G}}
\newcommand{\g}{\mathfrak{g}}
\newcommand{\p}{\mathfrak{p}}
\newcommand{\m}{\mathfrak{m}}
\newcommand{\T}{\mathcal{T}}
\newcommand{\X}{\mathbf{X}}
\newcommand{\x}{\mathbf{x}}
\newcommand{\Ord}{\mathrm{Ord}}
%\newcommand{\b}{\mathfrak{b}}
\newcommand{\IFF}{\Longleftrightarrow}
\newcommand{\conc}{^\frown}
\newcommand{\onto}{\xrightarrow{\text{onto}}}
\newcommand{\inj}{\xrightarrow{\text{1-1}}}
\newcommand{\downmapsto}{%
           \mathrel{\raisebox{.1em}{%
							\rotatebox[origin=c]{-90}{$\mapsto$}}}}
\newcommand{\upmapsto}{%
           \mathrel{\raisebox{.08em}{%
							\rotatebox[origin=c]{90}{$\mapsto$}}}}           
\newcommand{\ndivides}{%
  \mathrel{\mkern.5mu % small adjustment
    % superimpose \nmid to \big|
    \ooalign{\hidewidth$\big|$\hidewidth\cr$\nmid$\cr}%
  }%
}
\newcommand*{\defeq}{\mathrel{\rlap{%
                     \raisebox{0.3ex}{$\cdot$}}%
                     \raisebox{-0.3ex}{$\cdot$}}%
                     =}

\renewcommand{\epsilon}{\varepsilon}
\renewcommand{\phi}{\varphi}
\renewcommand{\H}{\mathcal{H}}
\renewcommand{\S}{\mathcal{S}}
\renewcommand{\O}{\mathcal{O}}
\renewcommand{\P}{\mathbb{P}}
\renewcommand{\u}{\mathbf{u}}
\renewcommand{\L}{\mathcal{L}}
\renewcommand{\iff}{\Leftrightarrow}
\newcommand{\forces}{\Vdash}





\title{Consistency results concerning $\pmb{\omega_1}$-trees}
%\subtitle{Risultati di consistenza per alberi di altezza $\omega_1$}
\author{Andrea Gadotti \hfill Relatore: Prof. Matteo Viale}
\date{14 ottobre 2016 \hfill Correlatore: Prof. Sy Friedman}
\institute{Università di Torino}
\titlegraphic{\hfill\includegraphics[height=1.5cm]{logo.png}}



\begin{document}

%%% diminuisce spazi verticali per displaymode %%%
\setlength{\abovedisplayskip}{1pt}
\setlength{\belowdisplayskip}{1pt}
\setlength{\abovedisplayshortskip}{1pt}
\setlength{\belowdisplayshortskip}{1pt}

\maketitle


\begin{frame}{Indice}
\setbeamertemplate{section in toc}[sections numbered]
\tableofcontents[hideallsubsections]
\end{frame}

\section{Alberi}

\begin{frame}{Alberi}

\begin{definition}
Un \emph{albero} è un insieme parzialmente ordinato $(T,<)$ tale che, per ogni $x \in T$, l'insieme
\begin{center}
$\down x \defeq \{y \in T : y < x\}$
\end{center}
è bene ordinato da $<$. Gli elementi di $T$ vengono detti \emph{nodi}. 
\end{definition}

\end{frame}


\begin{frame}{Alberi}

\vspace{10pt}

\begin{overprint}

\onslide<1>
\begin{center}
\begin{forest}
 for tree={grow=north}
	[$\bullet$, 
 		[$\bullet$, 
 			[$\bullet$, [$\bullet$][$\bullet$][$\bullet$]]
 			[$\bullet$]
 			[$\bullet$, [$\bullet$][$\bullet$]]
 			[$\bullet$]
 		]
 		[$\bullet$, 
 			[$\bullet$]
 			[$\bullet$, [$\bullet$, [$\bullet$][$\bullet$]]]
 			[$\bullet$]
 		]
 		[$\bullet$, 
 			[$\bullet$]
 			[$\bullet$, 
 				[$\bullet$, [$\bullet$, [$\bullet$][$\bullet$]][$\bullet$]][$\bullet$]
 			]
 		]
	]
\end{forest}
\end{center}


\onslide<2>
\begin{center}
\begin{forest}
 for tree={grow=north}
	[$\bullet$, 
 		[$\bullet$, 
 			[$\bullet$, [$\bullet$][$\bullet$][$\bullet$]]
 			[$\bullet$]
 			[$\bullet$, [$\bullet$][$\bullet$]]
 			[$\bullet$]
 		]
 		[$\bullet$, 
 			[$\bullet$]
 			[$\bullet$, [$\bullet$, [$\bullet$][$\bullet$]]]
 			[$\bullet$]
 		]
 		[$\bullet$, 
 			[$\bullet$]
 			[$\bullet$, 
 				[$\bullet$, [$\bullet$, fill=red [$\bullet$][$\bullet$]][$\bullet$]][$\bullet$]
 			]
 		]
	]
\end{forest}
\end{center}
\scalebox{2}{\textcolor{red}{$x$}}


\onslide<3>
\begin{center}
\begin{forest}
 for tree={grow=north}
	[$\bullet$, fill=blue, 
 		[$\bullet$, 
 			[$\bullet$, [$\bullet$][$\bullet$][$\bullet$]]
 			[$\bullet$]
 			[$\bullet$, [$\bullet$][$\bullet$]]
 			[$\bullet$]
 		]
 		[$\bullet$, 
 			[$\bullet$]
 			[$\bullet$, [$\bullet$, [$\bullet$][$\bullet$]]]
 			[$\bullet$]
 		]
 		[$\bullet$, fill=blue, 
 			[$\bullet$]
 			[$\bullet$, fill=blue, 
 				[$\bullet$, fill=blue, [$\bullet$, fill=red [$\bullet$][$\bullet$]][$\bullet$]][$\bullet$]
 			]
 		]
	]
\end{forest}
\end{center}
\scalebox{2}{\textcolor{red}{$x$}, \textcolor{blue}{${\downarrow} \, x$}}

\end{overprint}

\end{frame}


\begin{frame}{L'albero ${}^{< \omega} 2$}

L'insieme delle sequenze binarie finite è 
\[
{}^{< \omega} 2 \defeq \{s \mid s \colon n \to 2 \text{ per qualche } n < \omega\}.
\]
Se definiamo l'ordine su ${}^{< \omega} 2$ dato da $s < t \iff s \subsetneq t$, otteniamo un albero binario.

\visible<2->{
\begin{center}
\begin{forest}
 for tree={grow=north}
	[$\emptyset$, 
 		[1, 
 			[11,
 				[111, 
 					[,edge=dashed][,edge=dashed]
 				]
 				[110,
 					[,edge=dashed][,edge=dashed]
 				]
 			]
 			[10,
 				[101, 
 					[,edge=dashed][,edge=dashed]
 				]
 				[100,
 					[,edge=dashed][,edge=dashed]
 				]
 			]
 		]
 		[0, 
 			[01,
 				[011, 
 					[,edge=dashed][,edge=dashed]
 				]
 				[010,
 					[,edge=dashed][,edge=dashed]
 				]
 			]
 			[00,
 				[001, 
 					[,edge=dashed][,edge=dashed]
 				]
 				[000,
 					[,edge=dashed][,edge=dashed]
 				]
 			]
 		]
 	]
\end{forest}
\end{center}
}

\end{frame}



\begin{frame}{Nozioni di base}
Sia $T$ un albero.
\begin{itemize}
\item L'\emph{altezza} di $x \in T$ è l'ordinale $\type(\down x)$.
\item L'\emph{$\alpha$-esimo livello di $T$} è l'insieme degli elementi di $T$ che hanno altezza $\alpha$.
\item L'\emph{altezza di $T$} è il più piccolo ordinale $\gamma$ tale che l'altezza di ogni $x \in T$ è $< \gamma$.
\item Un \emph{ramo} è un sottoinsieme linearmente ordinato massimale di $T$. L'altezza di un ramo si definisce nello stesso modo.
\item Un ramo è \emph{cofinale in $T$} se ha la stessa altezza di $T$.
%\item $T|\alpha$ is the subset of $T$ which contains every element of order strictly less than $\alpha$, i.e.\ $T|\alpha \defeq \cup_{\xi < \alpha} U_\xi$. Obviously $T|\alpha$ has height $\alpha$ if $\alpha \leq \height(T)$.
%\item We say that a tree $(T_2,<_2)$ is an \emph{extension} of $(T_1,<_1)$ if ${<_1} = {<_2} \cap (T_1 \times T_1)$, an \emph{end-extension} if $T_1=T_2|\alpha$ for some $\alpha$.
\end{itemize}
\end{frame}


\begin{frame}{L'albero ${}^{< \omega} 2$}
\vspace{13pt}

\begin{overprint}
\onslide<1> In ${}^{< \omega} 2$, l'altezza di un nodo corrisponde alla sua lunghezza vista come sequenza. Perciò il livello $n$-esimo contiene tutte e sole le sequenze in ${}^{< \omega} 2$ che hanno lunghezza $n$. 

\onslide<2> Quindi ${}^{<\omega} 2$ ha altezza $\omega$ e tutti i suoi livelli sono insiemi finiti. Inoltre, $\{0^n : n < \omega\}$ è un ramo infinito (dove $0^n$ indica la sequenza $\underbrace{0000 \ldots}_{n}$).
\end{overprint}

\vspace{-10pt}

\begin{center}
\begin{forest}
 for tree={grow=north}
	[$\emptyset$, 
 		[1, 
 			[11,
 				[111, 
 					[,edge=dashed][,edge=dashed]
 				]
 				[110,
 					[,edge=dashed][,edge=dashed]
 				]
 			]
 			[10,
 				[101, 
 					[,edge=dashed][,edge=dashed]
 				]
 				[100,
 					[,edge=dashed][,edge=dashed]
 				]
 			]
 		]
 		[0, 
 			[01,
 				[011, 
 					[,edge=dashed][,edge=dashed]
 				]
 				[010,
 					[,edge=dashed][,edge=dashed]
 				]
 			]
 			[00,
 				[001, 
 					[,edge=dashed][,edge=dashed]
 				]
 				[000,
 					[,edge=dashed][,edge=dashed]
 				]
 			]
 		]
 	]
\end{forest}
\end{center}

\end{frame}


\section{La Tree Property}


\begin{frame}{Il Lemma di König}

\begin{lemma}[König, 1927]
Sia $T$ un albero di altezza $\omega$. Se i livelli di $T$ sono insiemi finiti, allora $T$ ha un ramo infinito.
\end{lemma}

\pause

\begin{alertblock}{Osservazione}
Possiamo sostituire ``$T$ ha un ramo infinito'' con ``$T$ ha un ramo di altezza $\omega$'' oppure ``$T$ ha un ramo cofinale''.
\end{alertblock}

\pause

\begin{exampleblock}{Domanda}
Possiamo generalizzare il lemma di König per cardinali maggiori di $\omega$?
\end{exampleblock}

\end{frame}

% mostrare che il lemma è chiaramente vero in 2^<omega


\begin{frame}{Un albero di altezza $\pmb{\omega_1}$}


\begin{overprint}
\onslide<1-2> Consideriamo l'albero $(T,<)$ dove
\[
T \defeq \{s \mid s \colon \alpha \to 2 \text{ con $\alpha < \omega_1$ e $s$ ha un numero finito di 1}\}
\]
e $<$ è come prima $\subsetneq$.

%\onslide<3> Consideriamo l'albero $(T,<)$ dove
%\[
%T \defeq \{s \mid s \colon \alpha \to 2 \text{ con $\alpha < \omega_1$ e $s$ ha un numero finito di 1}\}
%\]
%e $<$ è come prima $\subsetneq$. Con $0^\alpha$ indichiamo $\underbrace{00000000 \ldots}_{\alpha}$

\onslide<3>
\vspace{-8pt}
\begin{alertblock}{Osservazione}
$T$ ha altezza $\omega_1$ e i livelli di $T$ sono tutti numerabili. In più, $\{0^\alpha : \alpha < \omega_1\}$ è un ramo in $T$ di altezza $\omega_1$. Non tutti i rami hanno altezza $\omega_1$: $\{1^n : n < \omega\}$ è un ramo di altezza $\omega$.
\end{alertblock}
\end{overprint}

\vspace{-20pt}

\visible<2->{
\begin{center}
\begin{forest}
 for tree={grow=north}
	[$\emptyset$, 
 		[1, 
 			[11,
 				[111, 
 					[$\vdots$,edge=dashed [,edge=dashed]]
 				]
 				[110,
 					[$\vdots$,edge=dashed [,edge=dashed]]
 				]
 			]
 			[10,
 				[101, 
 					[$\vdots$,edge=dashed [,edge=dashed]]
 				]
 				[100,
 					[$\vdots$,edge=dashed [,edge=dashed]]
 				]
 			]
 		]
 		[0, 
 			[01,
 				[011, 
 					[$\vdots$,edge=dashed [,edge=dashed]]
 				]
 				[010,
 					[$\vdots$,edge=dashed [,edge=dashed]]
 				]
 			]
 			[00,
 				[001, 
 					[$\vdots$,edge=dashed [,edge=dashed]]
 				]
 				[000,
 					[$0^\alpha$,edge=dashed [,edge=dashed]]
 				]
 			]
 		]
 	]
\end{forest}
\end{center}
}
\end{frame}


\begin{frame}{La Tree Property}

\vspace{5pt}


Sia $\kappa$ un cardinale. Diciamo che $\kappa$ ha la \emph{tree property}, in simboli $\TP(\kappa)$, se è vero il seguente enunciato:
\begin{center}
\textit{Se $T$ è un albero di altezza $\kappa$ i cui livelli hanno taglia $< \kappa$, allora $T$ ha un ramo di altezza $\kappa$.}
\end{center}

\pause

\begin{alertblock}{Osservazione}
Il Lemma di König afferma che vale $\TP(\omega)$.
\end{alertblock}

\end{frame}


\begin{frame}{Gli alberi di Aronszajn}

\begin{definition}
Se $T$ è un controesempio per $\TP(\kappa)$, allora $T$ si dice \emph{$\kappa$-albero di Aronszajn}.
\end{definition}
Quindi $T$ è un $\kappa$-albero di Aronszajn se ha altezza $\kappa$ e livelli di cardinalità $< \kappa$, ma non ha nessun ramo di altezza $\kappa$.

\pause

\begin{theorem}[Aronszajn, 1934]
Esiste un $\omega_1$-albero di Aronszajn.
\end{theorem}

\end{frame}
% valutare se far vedere che esistono sempre k-alberi di aronszajn se k è singolare

\begin{frame}{Gli alberi di Aronszajn}

\begin{theorem}[Specker, 1949]
Sia $\kappa$ un cardinale infinito. Se $\kappa^{< \kappa} = \kappa$, allora esiste un $\kappa^+$-albero di Aronszajn.
\end{theorem}

\end{frame}


\begin{frame}{L'ipotesi del continuo}

\begin{exampleblock}{Domanda}
Esiste un insieme $X$ tale che $|\N| < |X| < |\R|$?
\end{exampleblock}

\pause

L'\emph{ipotesi del continuo}, in simboli \CH, è un'ipotesi formulata da Cantor e afferma che un tale insieme $X$ non esiste.

\pause

Nel 1940, Kurt Gödel dimostrò che nel sistema di assiomi \ZFC, \CH non può essere dimostrata falsa. Ovvero, \CH è \emph{consistente} con \ZFC.

\pause

Nel 1963, Paul Cohen introdusse la tecnica del \emph{forcing} per dimostrare che anche $\neg\CH$ è consistente con \ZFC. Quindi, complessivamente, \CH è \emph{indipendente} da \ZFC.

\end{frame}


\begin{frame}{L'ipotesi del continuo generalizzata}

L'\emph{ipotesi del continuo generalizzata}, in simboli \GCH, è il seguente enunciato:
\begin{center}
\textit{Sia $X$ un insieme infinito. Non esiste nessun insieme $Y$ tale che $|X| < |Y| < |{\PP(X)}|$.}
\end{center}

\pause

Anche \GCH è indipendente da \ZFC.

\end{frame}


\begin{frame}{Torniamo a $\TP(\kappa)$}

\vspace{10pt}

\begin{definition}
Un cardinale infinito $\kappa$ è \emph{regolare} se, per ogni famiglia $\F$ di insiemi di taglia $<\kappa$, $|\bigcup \F| = \kappa$ implica $|\F| = \kappa$. I cardinali infiniti non regolari si dicono \emph{singolari}.
\end{definition}

\pause

\begin{lemma}
Assumiamo che \GCH valga. Sia $\kappa$ un cardinale regolare. Allora $\kappa^{<\kappa} = \kappa$.
\end{lemma}

\pause

\begin{corollary}
Assumiamo che \GCH valga. Sia $\kappa$ un cardinale regolare. Allora esiste un $\kappa^+$-albero di Aronszajn.
\end{corollary}
La tree property per $\kappa$ successore di un cardinale singolare oppure in assenza di \GCH è tuttora oggeto di ricerca.

\end{frame}


\section{L'ipotesi di Suslin}


\begin{frame}{L'ipotesi di Suslin}

\begin{theorem}[Cantor, 1895]
Sia $(R,\prec)$ un insieme linearmente ordinato, denso, senza estremi, separabile e Dedekind completo. Allora $(R,\prec)$ è isomorfo a $(\R,<)$.
\end{theorem}

\pause

L'\emph{ipotesi di Suslin} (\SH) afferma che il teorema resta valido se sostituiamo l'ipotesi di separabilità con la più debole \emph{condizione della catena numerabile} (ccc): ogni collezione di intervalli aperti non vuoti e mutualmente disgiunti è numerabile.

\end{frame}


\begin{frame}{Le linee di Suslin}

\begin{definition}
Una \emph{linea di Suslin} è un insieme linearmente ordinato, denso e che soddisfa la ccc, ma non è separabile.
\end{definition}

\pause

Si verifica facilmente che per un insieme linearmente ordinato e denso, considerare il completamento di Dedekind ed aggiungere/eliminare estremi non influisce sulla proprietà di separabilità. Perciò:

\pause

\begin{block}{Fatto}
\textit{L'ipotesi di Suslin è vera se e solo se non esiste nessuna linea di Suslin.}
\end{block}

\end{frame}


\begin{frame}{Gli alberi di Suslin}

\begin{definition}
Un sottoinsieme $A$ di un albero si dice \emph{anticatena} se 
\[
\forall x,y \in A \ [x \nless y \ \wedge \ y \nless x].
\]
\end{definition}

\pause

\begin{definition}
Un \emph{albero di Suslin} è un albero di altezza $\omega_1$ tale che ogni ramo è numerabile ed ogni anticatena è numerabile.
\end{definition}

\pause

\begin{theorem}[Kurepa, 1935]
Esiste una linea di Suslin se e solo se esiste un albero di Suslin.
\end{theorem}

\end{frame}


\begin{frame}{L'indipendenza di \SH}

L'esistenza di alberi di Suslin non è dimostrabile né refutabile in \ZFC. Più precisamente:

\begin{theorem}[Tennenbaum, 1963]
C'è un modello di \ZFC in cui esiste un albero di Suslin.
\end{theorem}

\begin{theorem}[Solovay-Tennenbaum, 1971]
C'è un modello di \ZFC in cui non esiste nessun albero di Suslin.
\end{theorem}

Per entrambi i risultati venne utilizzata la tecnica del forcing.

\end{frame}


\begin{frame}{La consistenza di \SH: uno sguardo alla dimostrazione}

Supponiamo che $T$ sia un albero di Suslin (in $V$). Osserviamo che capovolgendo $T$ ``all'ingiù'', otteniamo un insieme di condizioni per il forcing. Inoltre:
\begin{itemize}
\item  Poiché ogni anticatena in $T$ è numerabile, tale insieme insieme ha la ccc (nel senso del forcing).
\item Ogni filtro generico $G$ per $T$ è un ramo di altezza $\omega_1$. Ovvero, in $V[G]$, $T$ ha un ramo cofinale.
\end{itemize}

Quindi, se $T$ è un albero di Suslin e prendiamo $T$ come insieme di condizioni per il forcing, otteniamo che in ogni estensione generica $T$ non è più un albero di Suslin. Ovvero, per annullare la ``suslinità'' di un $T$ fissato, possiamo fare forcing con $T$ stesso.

\end{frame}


\begin{frame}{La consistenza di \SH: uno sguardo alla dimostrazione}

\begin{alertblock}{Problema}
Il forcing appena descritto ``annienta'' un determinato albero di Suslin, ma può succedere che allo stesso tempo ne produca di nuovi.
\end{alertblock}

Per risolvere questa complicazione serve un qualche tipo di iterazione. Un approccio naïve rischia però di non preservare i cardinali. Il problema venne definitivamente risolto da Solovay e Tennenbaum nel 1965, che introdussero il \emph{forcing iterato} e lo usarono per dimostrare la consistenza di \SH.

\end{frame}


\section{L'ipotesi di Kurepa}


\begin{frame}{Gli alberi di Kurepa}

\begin{definition}
Un \emph{albero di Kurepa} è un albero di altezza $\omega_1$ tale che ogni livello è numerabile e in cui ci sono almeno $\aleph_2$ rami cofinali distinti.
\end{definition}

L'\emph{ipotesi di Kurepa} asserisce che esistono alberi di Kurepa.

\pause

L'ipotesi di Kurepa è consistente con \ZFC:

\begin{theorem}[Stewart, 1966]
C'è un modello di \ZFC in cui esiste un albero di Kurepa.
\end{theorem}

\end{frame}


\begin{frame}{L'ipotesi di Kurepa e i cardinali inaccessibili}

\vspace{5pt}

\begin{definition}
Un cardinale $\kappa > \omega$ si dice \emph{inaccessibile} se è regolare e $2^\lambda < \kappa$ per ogni cardinale $\lambda < \kappa$.
\end{definition}

\begin{block}{Fatto}
\textit{\ZFC non può dimostrare che l'esistenza di un cardinale inaccessibile è consistente con \ZFC.}
\end{block}

\pause

\begin{theorem}[Silver, 1971]
Assumiamo che esista un cardinale inaccessibile. Allora c'è un modello di \ZFC in cui non esistono alberi di Kurepa.
\end{theorem}

Si dimostra che, se non esistono alberi di Kurepa, allora esiste un cardinale inaccessibile. Quindi l'esistenza di cardinali inaccessibili è \emph{equiconsistente} con la negazione dell'ipotesi di Kurepa.

\end{frame}


\section{Alberi in $L$}


\begin{frame}{L'universo costruibile di Gödel}

\begin{definition}
La \emph{gerarchia costruibile di Gödel} è la sequenza $\langle L_\alpha \mid \alpha \in \Ord \rangle$ definita ricorsivamente così:
\begin{align*}
L_0 &= \emptyset;\\
L_{\alpha+1} &= \operatorname{def}(L_\alpha);\\
L_{\eta} &= \bigcup_{\gamma < \eta} L_\gamma \quad \text{se $\eta$ è limite};
\end{align*}
dove $\operatorname{def}(L_\alpha)$ è la collezione dei sottoinsiemi di $L_\alpha$ definibili nella struttura $(L_\alpha,\in)$. L'\emph{universo costruibile di Gödel} è $L \defeq \bigcup_{\alpha \in \Ord} L_\alpha$.
\end{definition}

\begin{block}{Fatto}
\textit{L'enunciato ``$V=L$'' è consistente con \ZFC.}
\end{block}

\end{frame}


\begin{frame}{Alberi di Suslin in $L$}

\vspace{5pt}

\begin{definition}
Il \emph{principio del diamante} ($\diamondsuit$) afferma che esiste una successione $\langle S_\alpha : \alpha < \omega_1 \rangle$ dove $S_\alpha \sse \alpha$, tale che per ogni $X \sse \omega_1$, l'insieme $\{\alpha < \omega_1 : X \cap \alpha = S_\alpha \}$ è stazionario in $\omega_1$.
\end{definition}

\pause

\begin{theorem}[Jensen, 1971]
$\diamondsuit$ implica che esiste un albero di Suslin.
\end{theorem}

\pause

\begin{theorem}[Jensen, 1971]
$V=L \imp \diamondsuit$. In particolare, se $V=L$ allora esiste un albero di Suslin.
\end{theorem}

\pause

In realtà, Jensen dimostrò ``$V=L \imp$ esiste un albero di Suslin'' già nel 1968. Introdusse $\diamondsuit$ più tardi, identificandolo e isolandolo proprio a partire dalla dimostrazione di questo fatto.

\end{frame}


\begin{frame}{Alberi di Kurepa in $L$}

Ispirandosi al teorema di Jensen, qualche anno più tardi Solovay dimostrò che:

\begin{theorem}[Solovay, 1971]
Se $V=L$, allora esiste un albero di Kurepa.
\end{theorem}

\pause

Anche dalla dimostrazione di questo teorema venne estratto un principio più tecnico è generale: $\diamondsuit^+$.\\
\pause
$\diamondsuit^+$ implica che esiste un albero di Kurepa. Anche $\diamondsuit^+$ vale in $L$, ma è dimostrabilmente più forte rispetto a $\diamondsuit$.

\end{frame}


\section{Automorfismi di alberi}

\begin{frame}{Isomorfismi di alberi}

\begin{definition}
Consideriamo due alberi $(T_1,<_1)$ e $(T_2,<_2)$. Un \emph{isomorfismo} di $T_1$ con $T_2$ è un isomorfismo d'ordini, ovvero una biezione $\sigma \colon T_1 \to T_2$ tale che $x <_1 y \iff \sigma(x) <_2 \sigma(y)$. Un \emph{automorfismo} di un albero è un isomorfismo con se stesso.
\end{definition}

\pause

\begin{block}{Osservazioni}
Se $\sigma$ è un isomorfismo di $T_1$ con $T_2$, è immediato che:
\begin{itemize}
\item Per ogni $x \in T_1$, $x$ e $\sigma(x)$ hanno la stessa altezza.
\item $T_1$ e $T_2$ hanno la stessa altezza.
\item L'immagine di un ramo di altezza $\alpha$ è un ramo di altezza $\alpha$.
\end{itemize}
\end{block}

\end{frame}


\begin{frame}{Esempio: due alberi isomorfi}

\only<1>{
\footnotesize
\begin{center}
\begin{forest}
 for tree={grow=north, circle, draw, inner sep=2, outer sep=4}
	[A, 
 		[D, [O][N][M][L]]
 		[C, [I][H][G]]
 		[B, [F][E]]
	]
\end{forest}
\end{center}

\begin{center}
\small
\begin{forest}
 for tree={grow=north, circle, draw, inner sep=2.5, outer sep=4}
	[a, 
 		[d, [o][n][m][l]]
 		[c, [i][h]]
 		[b, [g][f][e]]
	]
\end{forest}
\end{center}
}


\only<2>{
\footnotesize
\begin{center}
\begin{forest}
 for tree={grow=north, circle, draw, inner sep=2, outer sep=4}
	[A, fill=cyan,
 		[D, for tree={fill=cyan, edge=cyan} [O][N][M][L]]
 		[C, for tree={fill=orange, edge=orange} [I][H][G]]
 		[B, for tree={fill=green, edge=green} [F][E]]
	]
\end{forest}
\end{center}

\begin{center}
\small
\begin{forest}
 for tree={grow=north, circle, draw, inner sep=2.5, outer sep=4}
	[a, fill=cyan,
 		[d, for tree={fill=cyan, edge=cyan} [o][n][m][l]]
 		[c, for tree={fill=green, edge=green} [i][h]]
 		[b, for tree={fill=orange, edge=orange} [g][f][e]]
	]
\end{forest}
\end{center}
}


%\begin{center}
%\begin{forest}
% for tree={grow=north}
%	[$\bullet$, text=blue
% 		[$\bullet$, edge=blue,text=blue [$\bullet$, edge=blue,text=blue][$\bullet$, edge=blue,text=blue][$\bullet$, edge=blue,text=blue][$\bullet$, edge=blue,text=blue]]
% 		[$\bullet$, edge=red,text=red, [$\bullet$, edge=red,text=red][$\bullet$, edge=red,text=red][$\bullet$, edge=red,text=red]]
% 		[$\bullet$, edge=green,text=green [$\bullet$, edge=green,text=green][$\bullet$, edge=green,text=green]]
%	]
%\end{forest}
%\end{center}
%
%\begin{center}
%\begin{forest}
% for tree={grow=north}
%	[$\bullet$, text=blue
% 		[$\bullet$, edge=blue,text=blue [$\bullet$, edge=blue,text=blue][$\bullet$, edge=blue,text=blue][$\bullet$, edge=blue,text=blue][$\bullet$, edge=blue,text=blue]]
% 		[$\bullet$, edge=green,text=green [$\bullet$, edge=green,text=green][$\bullet$, edge=green,text=green]]
% 		[$\bullet$, edge=red,text=red, [$\bullet$, edge=red,text=red][$\bullet$, edge=red,text=red][$\bullet$, edge=red,text=red]]
% 	]
%\end{forest}
%\end{center}

\end{frame}


\begin{frame}{Esempio: due alberi non isomorfi}

\only<1>{
\begin{center}
\begin{forest}
 for tree={grow=north}
	[$\bullet$, 
 		[$\bullet$, [$\bullet$][$\bullet$][$\bullet$][$\bullet$]]
 		[$\bullet$, [$\bullet$][$\bullet$][$\bullet$][$\bullet$]]
 		[$\bullet$, [$\bullet$]]
	]
\end{forest}
\end{center}

\begin{center}
\begin{forest}
 for tree={grow=north}
	[$\bullet$, 
 		[$\bullet$, [$\bullet$][$\bullet$][$\bullet$][$\bullet$]]
 		[$\bullet$, [$\bullet$][$\bullet$][$\bullet$]]
 		[$\bullet$, [$\bullet$][$\bullet$]]
	]
\end{forest}
\end{center}
}

\only<2>{
\begin{center}
\begin{forest}
 for tree={grow=north}
	[$\bullet$, 
 		[$\bullet$, [$\bullet$][$\bullet$][$\bullet$][$\bullet$]]
 		[$\bullet$, [$\bullet$][$\bullet$][$\bullet$][$\bullet$]]
 		[$\bullet$, fill=green, [$\bullet$]]
	]
\end{forest}
\end{center}

\begin{center}
\begin{forest}
 for tree={grow=north}
	[$\bullet$, 
 		[$\bullet$, fill=red, [$\bullet$][$\bullet$][$\bullet$][$\bullet$]]
 		[$\bullet$, fill=red, [$\bullet$][$\bullet$][$\bullet$]]
 		[$\bullet$, fill=red, [$\bullet$][$\bullet$]]
	]
\end{forest}
\end{center}
}

\end{frame}


\begin{frame}{Gli alberi normali}

\begin{definition}
Un albero $(T,<)$ di altezza $\alpha \leq \omega_1$ si dice \emph{normale} se:.
\begin{enumerate}[(i)]
\item $T$ ha un'unica radice;
\item ogni livello di $T$ è numerabile;
\item se $x$ non è massimale in $T$, allora ci sono esattamente due \emph{successori immediati} di $x$;
\item se $x \in T$ allora c'è un $y>x$ ad ogni livello superiore di $T$;
\item se $\eta < \alpha$ è un ordinale limite e $x,y \in \L_\eta^T$ sono tali che $\down x = \down y$, allora $x=y$.
\end{enumerate}
\end{definition}

\end{frame}


\begin{frame}{Un esempio di albero normale}

L'albero 
\[
T \defeq \{s \mid s \colon \alpha \to 2 \text{ con $\alpha < \omega_1$ e $s$ ha un numero finito di 1}\}
\]
considerato prima è normale.

\begin{center}
\begin{forest}
 for tree={grow=north}
	[$\emptyset$, 
 		[1, 
 			[11,
 				[111, 
 					[$\vdots$,edge=dashed [,edge=dashed]]
 				]
 				[110,
 					[$\vdots$,edge=dashed [,edge=dashed]]
 				]
 			]
 			[10,
 				[101, 
 					[$\vdots$,edge=dashed [,edge=dashed]]
 				]
 				[100,
 					[$\vdots$,edge=dashed [,edge=dashed]]
 				]
 			]
 		]
 		[0, 
 			[01,
 				[011, 
 					[$\vdots$,edge=dashed [,edge=dashed]]
 				]
 				[010,
 					[$\vdots$,edge=dashed [,edge=dashed]]
 				]
 			]
 			[00,
 				[001, 
 					[$\vdots$,edge=dashed [,edge=dashed]]
 				]
 				[000,
 					[$0^\alpha$,edge=dashed [,edge=dashed]]
 				]
 			]
 		]
 	]
\end{forest}
\end{center}

\end{frame}


\begin{frame}{Unicità degli alberi normali numerabili}

Se ``tagliamo'' l'albero precedente ad altezza $\alpha < \omega_1$, otteniamo banalmente un albero $T|\alpha$ di altezza $\alpha$. Tali $T|\alpha$ per $\alpha < \omega_1$ sono tutti normali. \pause Ma non solo: la famiglia $\{T|\alpha : \alpha < \omega_1\}$ esaurisce la classe degli alberi normali numerabili. Infatti:

\pause

\begin{theorem}
Siano $T_1$ e $T_2$ alberi normali di uguale altezza $\alpha < \omega_1$. Allora $T_1$ e $T_2$ sono isomorfi.
\end{theorem}

\pause

\begin{exampleblock}{Domanda}
È possibile generalizzare il teorema ad alberi normali di altezza $\omega_1$?
\end{exampleblock}

\end{frame}


\begin{frame}{Alberi normalizzati}

\begin{block}{Fatto}
\textit{Sia $T$ un albero di Aronszajn/Suslin/Kurepa. Allora esiste un albero di Aronszajn/Suslin/Kurepa che è normale.}
\end{block}

\pause

Ovviamente un albero di Aronszajn non può essere isomorfo a un albero di Kurepa, quindi la risposta alla domanda precedente è negativa.

\end{frame}


\begin{frame}{Alberi omogenei e alberi rigidi}

\begin{definition}
Un albero $T$ si dice \emph{omogeneo} se, per ogni $x,y \in T$ allo stesso livello, esiste un automorfismo $\sigma$ di $T$ tale che $\sigma(x)=y$ e $\sigma(y)=x$. Un albero si dice \emph{rigido} se il suo unico automorfismo è l'identità.
\end{definition}

\pause

Usando il fatto che tutti gli alberi normali numerabili della stessa altezza sono isomorfi, è possibile provare che:

\begin{block}{Fatto}
\textit{Tutti gli alberi normali numerabili sono omogenei.}
\end{block}

\end{frame}


\begin{frame}{Alberi di Suslin omogenei e rigidi}

\begin{theorem}[Jensen, 1971]
Se vale $\diamondsuit$, allora esiste un albero di Suslin normale e omogeneo.
\end{theorem}

\pause

\begin{theorem}[Jensen, 1971]
Se vale $\diamondsuit$, allora esiste un albero di Suslin normale e rigido.
\end{theorem}

\end{frame}


\begin{frame}[plain]
\begin{center}
\Large \textbf{Grazie per l'attenzione!}
\end{center}
\end{frame}








\end{document}






























