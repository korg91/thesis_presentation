\documentclass{beamer}

\uselanguage{italian}
\languagepath{italian}
\deftranslation[to=italian]{Theorem}{Teorema}
\deftranslation[to=italian]{Definition}{Definizione}
\deftranslation[to=italian]{Corollary}{Corollario}

\usetheme[progressbar=frametitle, block=fill]{metropolis} % Use metropolis theme

\usepackage[utf8]{inputenc}
\usepackage{forest}
\usepackage{xspace}
\usepackage{xcolor}

\usepackage{tikz-cd}
\usepackage{tikz}
\usetikzlibrary{calc} 
\usetikzlibrary{decorations.text}
\usetikzlibrary{decorations.pathreplacing}
\usetikzlibrary{arrows}

%%% inizio comandi per stile per teoremi: "numero. Titolo" %%%
\newtheoremstyle{num.custom-title}
  {\topsep}   % ABOVESPACE
  {\topsep}   % BELOWSPACE
  {\itshape}  % BODYFONT
  {0pt}       % INDENT (empty value is the same as 0pt)
  {\bfseries} % HEADFONT
  {}         % HEADPUNCT
  {5pt plus 1pt minus 1pt} % HEADSPACE
  {\thmnumber{#2.}\thmnote{ #3}}
  
\theoremstyle{num.custom-title}  
\newtheorem{teo_custom-title}[theorem]{} % per usarlo basta \begin{teo_custom-title}[<Titolo teorema>] (usa automaticamente la numerazione di [teo])
%%% fine comandi per stile per teoremi: "numero. Titolo" %%%

%%% inizio comandi per stile per teoremi: "Titolo" %%%
\newtheoremstyle{custom-title}{}{}{\normalfont}{}{\bfseries}{.}{.5em}{\thmnote{#3}#1}
\theoremstyle{custom-title}
\newtheorem*{teo_custom-title_nonum}{}
%%% fine comandi per stile per teoremi: "numero. Titolo" %%%

\newenvironment{claim}[1]{\par\noindent\underline{Claim#1:}\space}{} %per i claim
\newenvironment{claimproof}[1]{\par\noindent\underline{Proof:}\space#1}{\leavevmode\unskip\penalty9999 \hbox{}\nobreak\hfill\quad\hbox{$\blacksquare$}} %per le dimostrazioni dei claim

\DeclareMathOperator{\dom}{dom}
\DeclareMathOperator{\ran}{ran}
\DeclareMathOperator{\orb}{orb}
\DeclareMathOperator{\id}{id}
\DeclareMathOperator{\rk}{rk}
\DeclareMathOperator{\tor}{tor}
\let\o\relax % elimina \o dai comandi già definiti
\DeclareMathOperator{\o}{\mathsf{o}}
\let\Im\relax % elimina \o dai comandi già definiti
\DeclareMathOperator{\Im}{Im}
\DeclareMathOperator{\Zdv}{Zdv}
\DeclareMathOperator{\Hom}{Hom}
\DeclareMathOperator{\End}{End}
\DeclareMathOperator{\Ann}{Ann}
\DeclareMathOperator{\E}{\mathbb{E}}
\DeclareMathOperator{\PP}{\mathcal{P}}
\DeclareMathOperator{\LL}{\mathcal{L}}
\DeclareMathOperator{\Hrtg}{\text{Hrtg}}
\DeclareMathOperator{\J}{\mathcal{J}}
\DeclareMathOperator{\Z}{\mathbb{Z}}
\DeclareMathOperator{\U}{\mathfrak{U}}
\DeclareMathOperator{\PPP}{\mathbb{P}}
\DeclareMathOperator{\V}{\mathcal{V}}
\DeclareMathOperator{\Var}{Var}
\DeclareMathOperator{\Cov}{Cov}
\DeclareMathOperator{\a01}{\{0,1\}^{\star}}
\DeclareMathOperator{\imp}{\Rightarrow}
\DeclareMathOperator{\pmi}{\Leftarrow}
\DeclareMathOperator{\Pic}{Pic}
\DeclareMathOperator{\sm}{\setminus}
\DeclareMathOperator{\sse}{\subseteq}
\DeclareMathOperator{\cl}{cl}
\DeclareMathOperator{\Spec}{Spec}
\DeclareMathOperator{\Tr}{Tr}
\DeclareMathOperator{\spn}{span}
\DeclareMathOperator{\q}{\mathsf{q}}
\DeclareMathOperator{\h}{h}
\DeclareMathOperator{\GL}{GL}
\DeclareMathOperator{\type}{type}
\DeclareMathOperator{\height}{height}
\DeclareMathOperator{\length}{length}
\DeclareMathOperator{\restr}{\upharpoonright}
\DeclareMathOperator{\down}{\downarrow}
\DeclareMathOperator{\up}{\uparrow}
\DeclareMathOperator{\cf}{cf}
\DeclareMathOperator{\mos}{mos}
\DeclareMathOperator{\trcl}{trcl}
\DeclareMathOperator{\Fn}{Fn}
%\DeclareMathOperator{\conc}{^\frown}
%\DeclareMathOperator{\gcd}{GCD}


\newcommand{\AC}{\ensuremath{\mathsf{AC}}\xspace}
\newcommand{\CC}{\ensuremath{\mathsf{CC}}\xspace}
\newcommand{\DC}{\ensuremath{\mathsf{DC}}\xspace}
\newcommand{\ZF}{\ensuremath{\mathsf{ZF}}\xspace}
\newcommand{\ZFC}{\ensuremath{\mathsf{ZFC}}\xspace}
\newcommand{\LS}{\ensuremath{\mathsf{LS}}\xspace}
\newcommand{\AMC}{\ensuremath{\mathsf{AMC}}\xspace}
\newcommand{\TP}{\ensuremath{\mathsf{TP}}\xspace}
\newcommand{\GCH}{\ensuremath{\mathsf{GCH}}\xspace}
\newcommand{\CH}{\ensuremath{\mathsf{CH}}\xspace}
\newcommand{\SH}{\ensuremath{\mathsf{SH}}\xspace}
\newcommand{\nSH}{\ensuremath{\neg\mathsf{SH}}\xspace}
\newcommand{\MA}{\ensuremath{\mathsf{MA}}\xspace}
\newcommand{\ST}{\ensuremath{\mathsf{ST}}\xspace}
\newcommand{\KT}{\ensuremath{\mathsf{KT}}\xspace}
\newcommand{\KH}{\ensuremath{\mathsf{KH}}\xspace}
\newcommand{\HRule}{\rule{\linewidth}{0.5mm}} %per la prima pagina
\newcommand{\qedblack}{\hfill $\blacksquare$}
\newcommand{\ol}{\overline}
\newcommand{\ul}{\underline}
\newcommand{\A}{\mathcal{A}}
\newcommand{\B}{\mathcal{B}}
\newcommand{\C}{\mathbb{C}}
\newcommand{\F}{\mathcal{F}}
\newcommand{\I}{\mathcal{I}}
\newcommand{\M}{\mathcal{M}}
\newcommand{\Q}{\mathbb{Q}}
\newcommand{\N}{\mathbb{N}}
\newcommand{\R}{\mathbb{R}}
\newcommand{\G}{\mathcal{G}}
\newcommand{\g}{\mathfrak{g}}
\newcommand{\p}{\mathfrak{p}}
\newcommand{\m}{\mathfrak{m}}
\newcommand{\T}{\mathcal{T}}
\newcommand{\X}{\mathbf{X}}
\newcommand{\x}{\mathbf{x}}
\newcommand{\Ord}{\mathrm{Ord}}
%\newcommand{\b}{\mathfrak{b}}
\newcommand{\IFF}{\Longleftrightarrow}
\newcommand{\conc}{^\frown}
\newcommand{\onto}{\xrightarrow{\text{onto}}}
\newcommand{\inj}{\xrightarrow{\text{1-1}}}
\newcommand{\downmapsto}{%
           \mathrel{\raisebox{.1em}{%
							\rotatebox[origin=c]{-90}{$\mapsto$}}}}
\newcommand{\upmapsto}{%
           \mathrel{\raisebox{.08em}{%
							\rotatebox[origin=c]{90}{$\mapsto$}}}}           
\newcommand{\ndivides}{%
  \mathrel{\mkern.5mu % small adjustment
    % superimpose \nmid to \big|
    \ooalign{\hidewidth$\big|$\hidewidth\cr$\nmid$\cr}%
  }%
}
\newcommand*{\defeq}{\mathrel{\rlap{%
                     \raisebox{0.3ex}{$\cdot$}}%
                     \raisebox{-0.3ex}{$\cdot$}}%
                     =}

\renewcommand{\epsilon}{\varepsilon}
\renewcommand{\phi}{\varphi}
\renewcommand{\H}{\mathcal{H}}
\renewcommand{\S}{\mathcal{S}}
\renewcommand{\O}{\mathcal{O}}
\renewcommand{\P}{\mathbb{P}}
\renewcommand{\u}{\mathbf{u}}
\renewcommand{\L}{\mathcal{L}}
\renewcommand{\iff}{\Leftrightarrow}
\newcommand{\forces}{\Vdash}

\renewcommand{\emph}[1]{\textbf{#1}}




\title{Consistency results concerning $\pmb{\omega_1}$-trees}
%\subtitle{Risultati di consistenza per alberi di altezza $\omega_1$}
\author{Andrea Gadotti \hfill Relatore: Prof. Matteo Viale}
\date{14 ottobre 2016 \hfill Correlatore: Prof. Sy Friedman}
\institute{Università di Torino}
\titlegraphic{\hfill\includegraphics[height=1.5cm]{logo.png}}



\begin{document}

%%% diminuisce spazi verticali per displaymode %%%
\setlength{\abovedisplayskip}{1pt}
\setlength{\belowdisplayskip}{1pt}
\setlength{\abovedisplayshortskip}{1pt}
\setlength{\belowdisplayshortskip}{1pt}

\maketitle


\begin{frame}{Indice}
\setbeamertemplate{section in toc}[sections numbered]
\tableofcontents[hideallsubsections]
\end{frame}


\section{L'ipotesi di Suslin}

\begin{frame}{Alcune proprietà elementari per gli ordini}

Consideriamo un ordine parziale $(P,<)$.

\begin{definition}
$(P,<)$ è un ordine \emph{lineare} se ogni due elementi di $P$ sono confrontabili, ovvero $\forall x,y \in P \ [x<y \vee y<x \vee x=y]$.
\end{definition}

\begin{definition}
$(P,<)$ è \emph{denso} se, per ogni $x,y \in P$ tali che $x<y$, esiste $z \in P$ tale che $x < z < y$.
\end{definition}

\begin{definition}
$(P,<)$ è \emph{Dedekind-completo} se ogni sottoinsieme superiormente limitato ha un estremo superiore in $P$.
\end{definition}

\end{frame}


\begin{frame}{La condizione della catena numerabile}

Definiamo ora una proprietà meno nota:

\pause

\begin{definition}
Diciamo che un ordine parziale $(P,<)$ soddisfa la \emph{condizione della catena numerabile} (ccc) se ogni famiglia di intervalli aperti mutualmente disgiunti è numerabile.
\end{definition}

\pause

È noto che la retta reale $(\R,<)$ con l'ordine canonico è un ordine lineare, denso e Dedekind-completo. Inoltre, $(\R,<)$ soddisfa anche la ccc.

\begin{overprint}

\onslide<4>
\begin{center}
\begin{tikzpicture}[scale=8]
\draw[<->, thick] (-0.15,0) -- (1.1,0);

\draw[(-, ultra thick, blue] (-0.03,0) -- (-0.14,0);
\draw[(-), ultra thick, blue] (0.05,0) -- (0.15,0);
\draw[(-), ultra thick, blue] (0.20,0) -- (0.25,0);
\draw[(-), ultra thick, blue] (0.32,0) -- (0.6,0);
\draw[(-), ultra thick, blue] (0.7,0) -- (0.8,0);
\draw[(-, ultra thick, blue] (0.88,0) -- (1.09,0);
\draw (-0.2,0) node {\scalebox{1.3}{$\R:$}};
\end{tikzpicture}
\end{center}

\onslide<5>
\begin{center}
\begin{tikzpicture}[scale=8]
\draw[<->, thick] (-0.15,0) -- (1.1,0);

\draw[(-, ultra thick, blue] (-0.03,0) -- (-0.14,0);
\draw[(-), ultra thick, blue] (0.05,0) -- (0.15,0);
\draw[(-), ultra thick, blue] (0.20,0) -- (0.25,0);
\draw[(-), ultra thick, blue] (0.32,0) -- (0.6,0);
\draw[(-), ultra thick, blue] (0.7,0) -- (0.8,0);
\draw[(-, ultra thick, blue] (0.88,0) -- (1.09,0);
\draw (-0.2,0) node {\scalebox{1.3}{$\R:$}};

\coordinate (a) at (0.01,0);
\coordinate (b) at (0.175,0);
\coordinate (c) at (0.285,0);
\coordinate (d) at (0.65,0);
\coordinate (e) at (0.84,0);
\fill [red] (a) circle (.45pt);
\fill [red] (b) circle (.45pt);
\fill [red] (c) circle (.45pt);
\fill [red] (d) circle (.45pt);
\fill [red] (e) circle (.45pt);

\coordinate (f) at (-0.225,-0.13);
\fill [red] (f) circle (.45pt);
\draw (-0.13,-0.13) node {\scalebox{1.3}{$\in \Q$}};
\end{tikzpicture}
\end{center}

\end{overprint}

\end{frame}


\begin{frame}{La retta reale è separabile}

\begin{alertblock}{Osservazione}
Nell'argomento rappresentato graficamente, abbiamo in realtà usato il fatto che $\R$ è \emph{separabile}, ovvero ammette un sottoinsieme numerabile la cui chiusura topologica è tutto $\R$ (ad esempio, $\Q$ è un tale sottoinsieme).
\end{alertblock}

\end{frame}


\begin{frame}{Caratterizzazione della retta reale come ordine}

La retta reale è caratterizzabile in termini di queste proprietà:

\begin{theorem}[Cantor, 1895]
Sia $(R,\prec)$ un insieme linearmente ordinato, denso, senza estremi, separabile e Dedekind-completo. Allora $(R,\prec)$ è isomorfo a $(\R,<)$.
\end{theorem}

\pause

L'\emph{ipotesi di Suslin} (\SH), formulata da Suslin nel 1920, afferma che il teorema resta valido se sostituiamo l'ipotesi di separabilità con la più debole ccc.

\pause

\begin{exampleblock}{Domanda}
L'ipotesi di Suslin è vera o falsa?
\end{exampleblock}

\pause 

\begin{alertblock}{Spoiler}
Nessuna delle due.
\end{alertblock}

\end{frame}


\section{Alberi ``bassi''}

%\begin{frame}
%
%\begin{definition}
%Un \emph{albero} è un insieme parzialmente ordinato $(T,<)$ tale che, per ogni $x \in T$, l'insieme
%\begin{center}
%$\down x \defeq \{y \in T : y < x\}$
%\end{center}
%è bene ordinato da $<$. Gli elementi di $T$ vengono detti \emph{nodi}. 
%\end{definition}
%
%\end{frame}

\begin{frame}{Alberi di altezza $\pmb{\leq \omega}$}

Diamo per il momento una definizione semplificata di albero:

\begin{definition}
Un \emph{albero} è un insieme parzialmente ordinato $(T,<)$ tale che, per ogni $x \in T$, l'insieme
\begin{center}
$\down x \defeq \{y \in T : y < x\}$
\end{center}
è finito e linearmente ordinato da $<$. Gli elementi di $T$ vengono detti \emph{nodi}. 
\end{definition}

\end{frame}


\begin{frame}{Alberi di altezza $\pmb{\leq \omega}$}

\vspace{10pt}

\begin{overprint}

\onslide<1>
\begin{center}
\begin{forest}
 for tree={grow=north}
	[$\bullet$, 
 		[$\bullet$, 
 			[$\bullet$, [$\bullet$][$\bullet$][$\bullet$]]
 			[$\bullet$]
 			[$\bullet$, [$\bullet$][$\bullet$]]
 			[$\bullet$]
 		]
 		[$\bullet$, 
 			[$\bullet$]
 			[$\bullet$, [$\bullet$, [$\bullet$][$\bullet$]]]
 			[$\bullet$]
 		]
 		[$\bullet$, 
 			[$\bullet$]
 			[$\bullet$, 
 				[$\bullet$, [$\bullet$, [$\bullet$][$\bullet$]][$\bullet$]][$\bullet$]
 			]
 		]
	]
\end{forest}
\end{center}


\onslide<2>
\begin{center}
\begin{forest}
 for tree={grow=north}
	[$\bullet$, 
 		[$\bullet$, 
 			[$\bullet$, [$\bullet$][$\bullet$][$\bullet$]]
 			[$\bullet$]
 			[$\bullet$, [$\bullet$][$\bullet$]]
 			[$\bullet$]
 		]
 		[$\bullet$, 
 			[$\bullet$]
 			[$\bullet$, [$\bullet$, [$\bullet$][$\bullet$]]]
 			[$\bullet$]
 		]
 		[$\bullet$, 
 			[$\bullet$]
 			[$\bullet$, 
 				[$\bullet$, [$\bullet$, fill=red [$\bullet$][$\bullet$]][$\bullet$]][$\bullet$]
 			]
 		]
	]
\end{forest}
\end{center}
\scalebox{2}{\textcolor{red}{$x$}}


\onslide<3>
\begin{center}
\begin{forest}
 for tree={grow=north}
	[$\bullet$, fill=blue, 
 		[$\bullet$, 
 			[$\bullet$, [$\bullet$][$\bullet$][$\bullet$]]
 			[$\bullet$]
 			[$\bullet$, [$\bullet$][$\bullet$]]
 			[$\bullet$]
 		]
 		[$\bullet$, 
 			[$\bullet$]
 			[$\bullet$, [$\bullet$, [$\bullet$][$\bullet$]]]
 			[$\bullet$]
 		]
 		[$\bullet$, fill=blue, 
 			[$\bullet$]
 			[$\bullet$, fill=blue, 
 				[$\bullet$, fill=blue, [$\bullet$, fill=red [$\bullet$][$\bullet$]][$\bullet$]][$\bullet$]
 			]
 		]
	]
\end{forest}
\end{center}
\scalebox{2}{\textcolor{red}{$x$}, \textcolor{blue}{${\downarrow} \, x$}}

\end{overprint}

\end{frame}


\begin{frame}{L'albero $\pmb{{}^{< \omega} 2}$}

Ricordiamo che $\omega$ indica semplicemente $\N$.

\pause

L'insieme delle sequenze binarie finite è 
\[
{}^{< \omega} 2 \defeq \{s \mid s \colon n \to 2 \text{ per qualche } n < \omega\}.
\]
Se definiamo l'ordine su ${}^{< \omega} 2$ stabilendo che $s < t$ se e solo se $s$ è un prefisso di $t$, otteniamo un albero binario.

\vspace{-7pt}

\begin{center}
\begin{forest}
 for tree={grow=north}
	[$\emptyset$, 
 		[1, 
 			[11,
 				[111, 
 					[,edge=dashed][,edge=dashed]
 				]
 				[110,
 					[,edge=dashed][,edge=dashed]
 				]
 			]
 			[10,
 				[101, 
 					[,edge=dashed][,edge=dashed]
 				]
 				[100,
 					[,edge=dashed][,edge=dashed]
 				]
 			]
 		]
 		[0, 
 			[01,
 				[011, 
 					[,edge=dashed][,edge=dashed]
 				]
 				[010,
 					[,edge=dashed][,edge=dashed]
 				]
 			]
 			[00,
 				[001, 
 					[,edge=dashed][,edge=dashed]
 				]
 				[000,
 					[,edge=dashed][,edge=dashed]
 				]
 			]
 		]
 	]
\end{forest}
\end{center}

\end{frame}



%\begin{frame}{Nozioni di base}
%Sia $T$ un albero.
%\begin{itemize}
%\item L'\emph{altezza} di $x \in T$ è l'ordinale $\type(\down x)$.
%\item L'\emph{$\alpha$-esimo livello di $T$} è l'insieme degli elementi di $T$ che hanno altezza $\alpha$.
%\item L'\emph{altezza di $T$} è il più piccolo ordinale $\gamma$ tale che l'altezza di ogni $x \in T$ è $< \gamma$.
%\item Un \emph{ramo} è un sottoinsieme linearmente ordinato massimale di $T$. L'altezza di un ramo si definisce nello stesso modo.
%\item Un ramo è \emph{cofinale in $T$} se ha la stessa altezza di $T$.
%%\item $T|\alpha$ is the subset of $T$ which contains every element of order strictly less than $\alpha$, i.e.\ $T|\alpha \defeq \cup_{\xi < \alpha} U_\xi$. Obviously $T|\alpha$ has height $\alpha$ if $\alpha \leq \height(T)$.
%%\item We say that a tree $(T_2,<_2)$ is an \emph{extension} of $(T_1,<_1)$ if ${<_1} = {<_2} \cap (T_1 \times T_1)$, an \emph{end-extension} if $T_1=T_2|\alpha$ for some $\alpha$.
%\end{itemize}
%\end{frame}


\begin{frame}{Nozioni di base}
Sia $T$ un albero. Possiamo definire in modo ovvio l'\emph{altezza} di un nodo in $T$ e l'\emph{$n$-esimo livello} di $T$. Inoltre, se l'altezza massima dei nodi in $T$ è $n$, allora diciamo che $T$ ha altezza $n+1$. Se i nodi in $T$ hanno altezza arbitrariamente grande, diciamo che $T$ ha altezza $\omega$, i.e.\ il primo cardinale infinito.

\begin{definition}
Un \emph{ramo} in $T$ è un sottoinsieme linearmente ordinato massimale di $T$. L'altezza di un ramo si definisce come quella degli alberi.
\end{definition}

\end{frame}


\begin{frame}{L'albero $\pmb{{}^{< \omega} 2}$}
\vspace{13pt}

\begin{overprint}
\onslide<1> In ${}^{< \omega} 2$, l'altezza di un nodo corrisponde alla sua lunghezza vista come sequenza. Perciò il livello $n$-esimo contiene tutte e sole le sequenze in ${}^{< \omega} 2$ che hanno lunghezza $n$. 

\onslide<2> Quindi ${}^{<\omega} 2$ ha altezza $\omega$ e tutti i suoi livelli sono insiemi finiti. Inoltre, $\{0^n : n < \omega\}$ è un ramo infinito (dove $0^n$ indica la sequenza $\underbrace{0000 \ldots}_{n}$).
\end{overprint}

\vspace{-10pt}

\begin{center}
\begin{forest}
 for tree={grow=north}
	[$\emptyset$, 
 		[1, 
 			[11,
 				[111, 
 					[,edge=dashed][,edge=dashed]
 				]
 				[110,
 					[,edge=dashed][,edge=dashed]
 				]
 			]
 			[10,
 				[101, 
 					[,edge=dashed][,edge=dashed]
 				]
 				[100,
 					[,edge=dashed][,edge=dashed]
 				]
 			]
 		]
 		[0, 
 			[01,
 				[011, 
 					[,edge=dashed][,edge=dashed]
 				]
 				[010,
 					[,edge=dashed][,edge=dashed]
 				]
 			]
 			[00,
 				[001, 
 					[,edge=dashed][,edge=dashed]
 				]
 				[000,
 					[,edge=dashed][,edge=dashed]
 				]
 			]
 		]
 	]
\end{forest}
\end{center}

\end{frame}


\begin{frame}{Il Lemma di König}

Le osservazioni fatte per ${}^{<\omega} 2$ valgono in generale:

\pause

\begin{lemma}[König, 1927]
Sia $T$ un albero di altezza $\omega$. Se i livelli di $T$ sono insiemi finiti, allora $T$ ha un ramo infinito.
\end{lemma}

\pause

\begin{alertblock}{Osservazione}
Possiamo sostituire ``$T$ ha un ramo infinito'' con ``$T$ ha un ramo di altezza $\omega$''.
\end{alertblock}

\pause

\begin{exampleblock}{Domanda}
Possiamo generalizzare il lemma di König per cardinali maggiori di $\omega$?
\end{exampleblock}

\end{frame}


\section{Isomorfismi di alberi}

\begin{frame}{Isomorfismi di alberi}

\begin{definition}
Consideriamo due alberi $(T_1,<_1)$ e $(T_2,<_2)$. Un \emph{isomorfismo} di $T_1$ con $T_2$ è un isomorfismo d'ordini, ovvero una biezione $\sigma \colon T_1 \to T_2$ tale che $x <_1 y \iff \sigma(x) <_2 \sigma(y)$.
\end{definition}

\pause

\begin{block}{Osservazioni}
Se $\sigma$ è un isomorfismo di $T_1$ con $T_2$, è immediato che:
\begin{itemize}
\item Per ogni $x \in T_1$, $x$ e $\sigma(x)$ hanno la stessa altezza.
\item $T_1$ e $T_2$ hanno la stessa altezza.
\item L'immagine di un ramo di altezza $\alpha$ è un ramo di altezza $\alpha$.
\end{itemize}
\end{block}

\end{frame}


\begin{frame}{Esempio: due alberi isomorfi}

\only<1>{
\footnotesize
\begin{center}
\begin{forest}
 for tree={grow=north, circle, draw, inner sep=2, outer sep=4}
	[A, 
 		[D, [O][N][M][L]]
 		[C, [I][H][G]]
 		[B, [F][E]]
	]
\end{forest}
\end{center}

\begin{center}
\small
\begin{forest}
 for tree={grow=north, circle, draw, inner sep=2.5, outer sep=4}
	[a, 
 		[d, [o][n][m][l]]
 		[c, [i][h]]
 		[b, [g][f][e]]
	]
\end{forest}
\end{center}
}


\only<2>{
\footnotesize
\begin{center}
\begin{forest}
 for tree={grow=north, circle, draw, inner sep=2, outer sep=4}
	[A, fill=cyan,
 		[D, for tree={fill=cyan, edge=cyan} [O][N][M][L]]
 		[C, for tree={fill=orange, edge=orange} [I][H][G]]
 		[B, for tree={fill=green, edge=green} [F][E]]
	]
\end{forest}
\end{center}

\begin{center}
\small
\begin{forest}
 for tree={grow=north, circle, draw, inner sep=2.5, outer sep=4}
	[a, fill=cyan,
 		[d, for tree={fill=cyan, edge=cyan} [o][n][m][l]]
 		[c, for tree={fill=green, edge=green} [i][h]]
 		[b, for tree={fill=orange, edge=orange} [g][f][e]]
	]
\end{forest}
\end{center}
}

\end{frame}


\begin{frame}{Esempio: due alberi non isomorfi}

\only<1>{
\begin{center}
\begin{forest}
 for tree={grow=north}
	[$\bullet$, 
 		[$\bullet$, [$\bullet$][$\bullet$][$\bullet$][$\bullet$]]
 		[$\bullet$, [$\bullet$][$\bullet$][$\bullet$][$\bullet$]]
 		[$\bullet$, [$\bullet$]]
	]
\end{forest}
\end{center}

\begin{center}
\begin{forest}
 for tree={grow=north}
	[$\bullet$, 
 		[$\bullet$, [$\bullet$][$\bullet$][$\bullet$][$\bullet$]]
 		[$\bullet$, [$\bullet$][$\bullet$][$\bullet$]]
 		[$\bullet$, [$\bullet$][$\bullet$]]
	]
\end{forest}
\end{center}
}

\only<2>{
\begin{center}
\begin{forest}
 for tree={grow=north}
	[$\bullet$, 
 		[$\bullet$, [$\bullet$][$\bullet$][$\bullet$][$\bullet$]]
 		[$\bullet$, [$\bullet$][$\bullet$][$\bullet$][$\bullet$]]
 		[$\bullet$, fill=green, [$\bullet$]]
	]
\end{forest}
\end{center}

\begin{center}
\begin{forest}
 for tree={grow=north}
	[$\bullet$, 
 		[$\bullet$, fill=red, [$\bullet$][$\bullet$][$\bullet$][$\bullet$]]
 		[$\bullet$, fill=red, [$\bullet$][$\bullet$][$\bullet$]]
 		[$\bullet$, fill=red, [$\bullet$][$\bullet$]]
	]
\end{forest}
\end{center}
}

\end{frame}


\begin{frame}{Gli alberi normali}

\begin{definition}
Un albero $(T,<)$ si dice \emph{normale} se:.
\begin{enumerate}[(i)]
\item $T$ ha un'unica radice;
\item ogni livello di $T$ è numerabile;
\item se $x$ non è massimale in $T$, allora ci sono esattamente due \emph{successori immediati} di $x$;
\item se $x \in T$ allora c'è un $y>x$ ad ogni livello superiore di $T$;
%\item se $\eta < \alpha$ è un ordinale limite e $x,y \in \L_\eta^T$ sono tali che $\down x = \down y$, allora $x=y$.
\end{enumerate}
\end{definition}

\end{frame}


\begin{frame}{$\pmb{{}^{< \omega} 2}$ è normale}

L'albero delle sequenze binarie finite ${}^{< \omega} 2$ è chiaramente normale.


\begin{center}
\begin{forest}
 for tree={grow=north}
	[$\emptyset$, 
 		[1, 
 			[11,
 				[111, 
 					[,edge=dashed][,edge=dashed]
 				]
 				[110,
 					[,edge=dashed][,edge=dashed]
 				]
 			]
 			[10,
 				[101, 
 					[,edge=dashed][,edge=dashed]
 				]
 				[100,
 					[,edge=dashed][,edge=dashed]
 				]
 			]
 		]
 		[0, 
 			[01,
 				[011, 
 					[,edge=dashed][,edge=dashed]
 				]
 				[010,
 					[,edge=dashed][,edge=dashed]
 				]
 			]
 			[00,
 				[001, 
 					[,edge=dashed][,edge=dashed]
 				]
 				[000,
 					[,edge=dashed][,edge=dashed]
 				]
 			]
 		]
 	]
\end{forest}
\end{center}

\end{frame}


\begin{frame}{Unicità degli alberi normali}

\pause

\begin{alertblock}{Osservazione}
È immediato dimostrare per induzione che due qualsiasi alberi normali (di altezza $\leq \omega$), se hanno la stessa altezza allora sono isomorfi.
\end{alertblock}

\pause

\begin{exampleblock}{Domanda}
Vale lo stesso anche per alberi più alti?
\end{exampleblock}

\end{frame}


\begin{frame}{Ricapitolando}
Abbiamo quindi tre domande a cui rispondere:
\begin{itemize}
\item[\textcolor{mLightGreen}{1)}] L'ipotesi di Suslin è vera?
\item[\textcolor{mLightGreen}{2)}] Il Lemma di König si può generalizzare a cardinali $>\omega$?
\item[\textcolor{mLightGreen}{3)}] Due alberi normali di uguale altezza $>\omega$ sono isomorfi?
\end{itemize}

\pause

La tesi si occupa di rispondere a queste domande, oltre a presentare alcuni dei risultati correlati più importanti. Descriveremo ora brevemente le risposte ai quesiti posti sopra.

\end{frame}


\section{Alberi}

\begin{frame}{Alberi}

Per trattare alberi di altezza arbitraria, è necessario innanzitutto generalizzare la definizione stessa di albero. Ricordiamo che:

\begin{definition}
Un ordine parziale $(P,<)$ è un \emph{buon ordine} se è un ordine lineare ed ogni sottoinsieme di $P$ ha un minimo.
\end{definition}

\pause

\begin{definition}
Un \emph{albero} è un insieme parzialmente ordinato $(T,<)$ tale che, per ogni $x \in T$, l'insieme
\[
\down x \defeq \{y \in T : y < x\}
\]
è bene ordinato da $<$.
\end{definition}

D'ora in poi, quando usiamo il termine albero ci riferiamo alla definizione appena data.

\end{frame}


\begin{frame}{Nozioni di base}

Consideriamo un albero $(T,<)$ e un nodo $x \in T$. Poiché $\down x$ è un insieme bene ordinato, c'è un ordinale isomorfo ad esso. Tale ordinale si dice \emph{tipo d'ordine} di $\down x$. Diciamo che $x$ \emph{ha altezza} $\alpha$ se il tipo d'ordine di $\down x$ è $\alpha$.

\pause

Le definizioni di livello e di altezza di un albero/ramo si generalizzano in modo immediato rispetto a quelle date in precedenza, usando però la nuova definizione di altezza di un nodo.

\end{frame}


\begin{frame}{Un albero di altezza $\pmb{\omega_1}$}

\begin{overprint}
\onslide<1> Il cardinale $\omega_1$ è il più piccolo ordinale che non si inietta in $\omega$.
\onslide<2-3> Il cardinale $\omega_1$ è il più piccolo ordinale che non si inietta in $\omega$.\\
Consideriamo l'albero $(T,<)$ dove
\[
T \defeq \{s \mid s \colon \alpha \to 2 \text{ con $\alpha < \omega_1$ e $s$ ha un numero finito di 1}\}
\]
e $s<t$ se e solo se $s$ è un prefisso di $t$.

%\onslide<3> Consideriamo l'albero $(T,<)$ dove
%\[
%T \defeq \{s \mid s \colon \alpha \to 2 \text{ con $\alpha < \omega_1$ e $s$ ha un numero finito di 1}\}
%\]
%e $<$ è come prima $\subsetneq$. Con $0^\alpha$ indichiamo $\underbrace{00000000 \ldots}_{\alpha}$

\onslide<4>
\vspace{-8pt}
\begin{alertblock}{Osservazione}
$T$ ha altezza $\omega_1$ e i livelli di $T$ sono tutti numerabili. In più, $\{0^\alpha : \alpha < \omega_1\}$ è un ramo in $T$ di altezza $\omega_1$. Non tutti i rami hanno altezza $\omega_1$: $\{1^n : n < \omega\}$ è un ramo di altezza $\omega$.
\end{alertblock}
\end{overprint}

\vspace{-20pt}

\visible<3->{
\begin{center}
\begin{forest}
 for tree={grow=north}
	[$\emptyset$, 
 		[1, 
 			[11,
 				[111, 
 					[$\vdots$,edge=dashed [,edge=dashed]]
 				]
 				[110,
 					[$\vdots$,edge=dashed [,edge=dashed]]
 				]
 			]
 			[10,
 				[101, 
 					[$\vdots$,edge=dashed [,edge=dashed]]
 				]
 				[100,
 					[$\vdots$,edge=dashed [,edge=dashed]]
 				]
 			]
 		]
 		[0, 
 			[01,
 				[011, 
 					[$\vdots$,edge=dashed [,edge=dashed]]
 				]
 				[010,
 					[$\vdots$,edge=dashed [,edge=dashed]]
 				]
 			]
 			[00,
 				[001, 
 					[$\vdots$,edge=dashed [,edge=dashed]]
 				]
 				[000,
 					[$0^\alpha$,edge=dashed [,edge=dashed]]
 				]
 			]
 		]
 	]
\end{forest}
\end{center}
}
\end{frame}


\section{La Tree Property}

\begin{frame}{Gli alberi di Aronszajn e la Tree Property}

\begin{lemma}[König]
Sia $T$ un albero di altezza $\omega$. Se i livelli di $T$ sono insiemi finiti, allora $T$ ha un ramo di altezza $\omega$.
\end{lemma}

\pause

\begin{definition}
Sia $\kappa$ un cardinale. $T$ è un \emph{$\kappa$-albero di Aronszajn} se ha altezza $\kappa$ e livelli di cardinalità $< \kappa$, ma non ha nessun ramo di altezza $\kappa$.
\end{definition}

\begin{alertblock}{Osservazione}
Il Lemma di König afferma che non ci sono $\omega$-alberi di Aronszajn.
\end{alertblock}

\pause

La \emph{tree property} per $\kappa$, in simboli $\TP(\kappa)$, è l'affermazione: 
\begin{center}
\textit{Non esistono $\kappa$-alberi di Aronszajn}.
\end{center}

\end{frame}


\begin{frame}{Gli alberi di Aronszajn}

Quindi $\TP(\kappa)$ generalizza il lemma di König ad un generico $\kappa$. \pause Ma $\TP(\kappa)$ non è vera per ogni $\kappa$:

\pause

\begin{theorem}[Aronszajn, 1934]
Esiste un $\omega_1$-albero di Aronszajn.
\end{theorem}

\pause

\begin{theorem}[Specker, 1949]
Sia $\kappa$ un cardinale infinito. Se $\kappa^{< \kappa} = \kappa$, allora esiste un $\kappa^+$-albero di Aronszajn.
\end{theorem}

\pause

La tree property in tutti gli altri casi è tuttora oggetto di ricerca.
\end{frame}


\section{L'indipendenza dell'ipotesi di Suslin}


\begin{frame}{Gli alberi di Suslin}

\begin{definition}
Un sottoinsieme $A$ di un albero si dice \emph{anticatena} se 
\[
\forall x,y \in A \ [x \nless y \ \wedge \ y \nless x].
\]
\end{definition}

Ogni livello è banalmente un'anticatena.

\pause

\begin{definition}
Un \emph{albero di Suslin} è un albero di altezza $\omega_1$ tale che ogni ramo è numerabile ed ogni anticatena è numerabile.
\end{definition}

\pause

\begin{theorem}[Kurepa, 1935]
L'ipotesi di Suslin è vera se e solo se non esiste nessun albero di Suslin.
\end{theorem}

\end{frame}


\begin{frame}{L'ipotesi del continuo}

Se $X$ è un insieme, denotiamo con $|X|$ la sua cardinalità.

\begin{exampleblock}{Domanda}
Esiste un insieme $X$ tale che $|\N| < |X| < |\R|$?
\end{exampleblock}

\pause

L'\emph{ipotesi del continuo}, in simboli \CH, è un'ipotesi avanzata da Cantor nel 1878 e afferma che un tale insieme $X$ non esiste.

\pause

Nel 1940, Kurt Gödel dimostrò che nel sistema di assiomi \ZFC, \CH non può essere dimostrata falsa. Ovvero, \CH è \emph{consistente} con \ZFC.

\pause

Nel 1963, Paul Cohen introdusse la tecnica del \emph{forcing} per dimostrare che anche $\neg\CH$ è consistente con \ZFC. Quindi, complessivamente, \CH è \emph{indipendente} da \ZFC.

\end{frame}


\begin{frame}{L'indipendenza di \SH}

L'esistenza di alberi di Suslin non è dimostrabile né refutabile in \ZFC. Più precisamente:

\begin{theorem}[Tennenbaum, 1963]
C'è un modello di \ZFC in cui esiste un albero di Suslin.
\end{theorem}

\begin{theorem}[Solovay-Tennenbaum, 1971]
C'è un modello di \ZFC in cui non esiste nessun albero di Suslin.
\end{theorem}

Per entrambi i risultati venne utilizzata la tecnica del forcing.

\end{frame}


\begin{frame}{La consistenza di \SH: uno sguardo alla dimostrazione}

Supponiamo che $T$ sia un albero di Suslin (in $V$). Osserviamo che capovolgendo $T$ ``all'ingiù'', otteniamo un insieme di condizioni per il forcing. Inoltre:
\begin{itemize}
\item  Poiché ogni anticatena in $T$ è numerabile, tale insieme ha la ccc (nel senso del forcing).
\item Ogni filtro generico $G$ per $T$ è un ramo di altezza $\omega_1$. Ovvero, in $V[G]$, $T$ ha un ramo cofinale.
\end{itemize}

Quindi, se $T$ è un albero di Suslin e prendiamo $T$ come insieme di condizioni per il forcing, otteniamo che in ogni estensione generica $T$ non è più un albero di Suslin. Ovvero, per annullare la ``suslinità'' di un $T$ fissato, possiamo fare forcing con $T$ stesso.

\end{frame}


\begin{frame}{La consistenza di \SH: uno sguardo alla dimostrazione}

\begin{alertblock}{Problema}
Il forcing appena descritto ``annienta'' un determinato albero di Suslin, ma può succedere che allo stesso tempo ne produca di nuovi.
\end{alertblock}

Per risolvere questa complicazione serve un qualche tipo di iterazione. Un approccio naïve rischia però di non preservare i cardinali. Il problema venne definitivamente risolto da Solovay e Tennenbaum nel 1965, che introdussero il \emph{forcing iterato} e lo usarono per dimostrare la consistenza di \SH.

\end{frame}


\section{Isomorfismi e automorfismi}

\begin{frame}{Gli alberi normali}

Generalizziamo la nozione di albero normale ad alberi più alti:

\begin{definition}
Un albero $(T,<)$ di altezza $\alpha \leq \omega_1$ si dice \emph{normale} se:.
\begin{enumerate}[(i)]
\item $T$ ha un'unica radice;
\item ogni livello di $T$ è numerabile;
\item se $x$ non è massimale in $T$, allora ci sono esattamente due \emph{successori immediati} di $x$;
\item se $x \in T$ allora c'è un $y>x$ ad ogni livello superiore di $T$;
\item[\textcolor{red}{(v)}] se $\eta < \alpha$ è un ordinale limite e $x,y \in \L_\eta^T$ sono tali che $\down x = \down y$, allora $x=y$.
\end{enumerate}
\end{definition}

\end{frame}


\begin{frame}{Un esempio di albero normale}

L'albero 
\[
T \defeq \{s \mid s \colon \alpha \to 2 \text{ con $\alpha < \omega_1$ e $s$ ha un numero finito di 1}\}
\]
considerato prima è normale.

\begin{center}
\begin{forest}
 for tree={grow=north}
	[$\emptyset$, 
 		[1, 
 			[11,
 				[111, 
 					[$\vdots$,edge=dashed [,edge=dashed]]
 				]
 				[110,
 					[$\vdots$,edge=dashed [,edge=dashed]]
 				]
 			]
 			[10,
 				[101, 
 					[$\vdots$,edge=dashed [,edge=dashed]]
 				]
 				[100,
 					[$\vdots$,edge=dashed [,edge=dashed]]
 				]
 			]
 		]
 		[0, 
 			[01,
 				[011, 
 					[$\vdots$,edge=dashed [,edge=dashed]]
 				]
 				[010,
 					[$\vdots$,edge=dashed [,edge=dashed]]
 				]
 			]
 			[00,
 				[001, 
 					[$\vdots$,edge=dashed [,edge=dashed]]
 				]
 				[000,
 					[$0^\alpha$,edge=dashed [,edge=dashed]]
 				]
 			]
 		]
 	]
\end{forest}
\end{center}

\end{frame}


\begin{frame}{Unicità degli alberi normali numerabili}

Se ``tagliamo'' l'albero precedente ad altezza $\alpha < \omega_1$, otteniamo banalmente un albero $T|\alpha$ di altezza $\alpha$. Tali $T|\alpha$ per $\alpha < \omega_1$ sono tutti normali. \pause Ma non solo: la famiglia $\{T|\alpha : \alpha < \omega_1\}$ esaurisce la classe degli alberi normali numerabili. Infatti:

\pause

\begin{theorem}
Siano $T_1$ e $T_2$ alberi normali di uguale altezza $\alpha < \omega_1$. Allora $T_1$ e $T_2$ sono isomorfi.
\end{theorem}

\pause

\begin{exampleblock}{Domanda}
È possibile generalizzare ulteriormente il teorema ad alberi normali di altezza $\omega_1$?
\end{exampleblock}

\end{frame}


\begin{frame}{Alberi normalizzati}

\begin{block}{Fatto}
\textit{Esiste un $\omega_1$-albero di Aronszajn che è normale.}
\end{block}

\pause

Ovviamente un albero di Aronszajn non può essere isomorfo a un albero che ha rami di altezza $\omega_1$.

\end{frame}


\begin{frame}{Un albero normale con rami di altezza $\omega_1$}

Ma l'albero
\[
T \defeq \{s \mid s \colon \alpha \to 2 \text{ con $\alpha < \omega_1$ e $s$ ha un numero finito di 1}\}
\]
considerato prima è normale e ha rami di altezza $\omega_1$!

\begin{center}
\begin{forest}
 for tree={grow=north}
	[$\emptyset$, 
 		[1, 
 			[11,
 				[111, 
 					[$\vdots$,edge=dashed [,edge=dashed]]
 				]
 				[110,
 					[$\vdots$,edge=dashed [,edge=dashed]]
 				]
 			]
 			[10,
 				[101, 
 					[$\vdots$,edge=dashed [,edge=dashed]]
 				]
 				[100,
 					[$\vdots$,edge=dashed [,edge=dashed]]
 				]
 			]
 		]
 		[0, 
 			[01,
 				[011, 
 					[$\vdots$,edge=dashed [,edge=dashed]]
 				]
 				[010,
 					[$\vdots$,edge=dashed [,edge=dashed]]
 				]
 			]
 			[00,
 				[001, 
 					[$\vdots$,edge=dashed [,edge=dashed]]
 				]
 				[000,
 					[$0^\alpha$,edge=dashed [,edge=dashed]]
 				]
 			]
 		]
 	]
\end{forest}
\end{center}

\end{frame}


\begin{frame}{Un albero normale con rami di altezza $\omega_1$}
Quindi esistono alberi normali di altezza $\omega_1$ che non sono isomorfi.
\end{frame}


\begin{frame}{Alberi omogenei e alberi rigidi}

\begin{definition}
Un albero $T$ si dice \emph{omogeneo} se, per ogni $x,y \in T$ allo stesso livello, esiste un automorfismo $\sigma$ di $T$ tale che $\sigma(x)=y$ e $\sigma(y)=x$. Un albero si dice \emph{rigido} se il suo unico automorfismo è l'identità.
\end{definition}

\pause

Usando il fatto che tutti gli alberi normali numerabili della stessa altezza sono isomorfi, è possibile provare che:

\begin{block}{Fatto}
\textit{Tutti gli alberi normali numerabili sono omogenei.}
\end{block}

\end{frame}


\begin{frame}{Alberi di Suslin omogenei e rigidi}

\begin{theorem}[Jensen, 1971]
C'è un modello di \ZFC in cui esiste un albero di Suslin normale e omogeneo.
\end{theorem}

\pause

\begin{theorem}[Jensen, 1971]
C'è un modello di \ZFC in cui esiste un albero di Suslin normale e rigido.
\end{theorem}

\pause

Per dimostrare questi due risultati, Jensen non ricorse al forcing, ma produsse i due alberi di Suslin all'interno dell'universo costruibile di Gödel.

\end{frame}


\begin{frame}[plain]
\begin{center}
\Large \textbf{Grazie per l'attenzione!}
\end{center}
\end{frame}








\end{document}






























